\documentclass[12pt]{article}
\usepackage{hyperref}
\usepackage[T1]{fontenc}
\usepackage[utf8]{inputenc}
\usepackage{amsmath}
\usepackage{graphicx}
\usepackage{hyperref}
\usepackage{cleveref}
\usepackage{authblk}
\usepackage{natbib}
\usepackage{natbibspacing}
\usepackage[left=2cm, right=2cm]{geometry}
\usepackage{subfigure}
\usepackage{indentfirst}
\usepackage{minted}
\setlength{\parindent}{2em}
\usemintedstyle{default}

\newcommand{\icarus}{Icarus}

\title{\textbf{DOCUMENTATION}}
\author{Zhixuan Li}
\date{\today}

\begin{document}
\maketitle 

\section{INTRODUCTION}
The Lagrangian code for planet formation code (Lag-PFS) is an open-source simulation 
code for simulating the disk systems, such as the evolution of disk and  
planets or satellites formation process in the protoplanetary disk (PPD) or 
circumplanetary disk (CPD) with Lagrangian method. This code is open-source on 
\href{https://github.com/Judithlll/CpdPhysics}{github}

This code is developed generally for every kinds of disk system, which allows 
users to implement the mechanisms they want, like the initial distribution 
of gas in disk, how the particles grow and drift, how the embryos 
migrate and accrete materials in disk and so on. And \textbf{NOTE} that because 
we firstly apply this code to simulate the satellites formation in CPDs, the 
discription in this documentation will use the 'planet' as the central body, while 
the satellites as orbital bodies in the system. 

The basic elements of disk system include the gas, solid particles, the central 
body, the objects in disk and also maybe icelines, gaps and so on.
The frame of this code is under the \textbf{main} folder, which contains 
how to conbine the elements of disk and run the system. And under 
\textbf{Test} folder, there are some sub-folders named by the users, which 
allow users to discribe the physical processes in there own ways. Under 
\textbf{Test/simple}, there's a simpliest example. Users can enter the 
test folder created by themselves:
\begin{verbatim}

        cd ~/CpdPhysics/Test/zxl_test

\end{verbatim}
and then run:
\begin{verbatim}

        python3 ../../main/runcpd.py 

\end{verbatim}

If you see: 
\begin{verbatim}
        [runcpd]: Congrats!! You finished a sucessful run which consume 1.17 seconds
        [runcpd]:finished
\end{verbatim}
in your terminal, you have sucessfully finished running your test. 

We will present the content and stucture in detail 
(sec. \ref{sec:a}), 
the acceleration of code: jump (sec. \ref{sec:b}).

\section{Content and Structure}
\label{sec:a}
We developed this code by python programming language and by object-oriented 
programming. We put all the relavant properties of one part in disk in a 
conrresponding class, so in the code you will see several classes below:
\begin{enumerate}
    \item Main classes:
        \begin{enumerate}
            \item core.Superparticles: the aggerates of lots of solid particles. 
            \item core.PLANET: include the properties of single orbital object in disk.
            \item core.ICELINE: include the properties of single iceline.
            \item gas.GAS: include the preperties distribution of gas.
            \item core.System: integrate every parts of disk, and discribe the interaction between them.
        \end{enumerate}
    \item Functional classes: some other classes like Single-SP, minTimes, and so on 
        are functional classes used in some medium steps, which have detailed 
        comments below them.
\end{enumerate}

Then, we will present you the structure of this code in the order of running, which 
is shown in \textbf{main/runcpd.py}, during which we will intraoduce you where and 
how to insert your own mechanisms. The overview of running process is shown in 
figure \ref{fig:cpdcode}
\begin{figure}[ht]
    \label{fig:cpdcode}
    \includegraphics* [width=\textwidth] {cpdcode.jpg}
    \caption{The process of running}
\end{figure}

\subsection {INITIALIZE}
Firstly, we initialize the system. The initialization is mainly done by the function 
\textbf{sim$\_$init} in \textbf{main/init.py}. 

The init has two options. One is running from scratch, the other is from 
the time point that last run ends, which need two pickle files under 
\textbf{/Test/user/pickles/}. Running from pickle files can be done by adding an 
arguement:
\begin{listing}[h]
    \label{code:b}
    \begin{minted}{python3}
        python3 ../../main/runcpd.py fromfile
    \end{minted}
\end{listing}

Let's now discuss the running from scratch. 
In the \textbf{sim$\_$init}, we firstly initialize an empty frame (system object), 
then we get the composition from .txt files under \textbf{config/(comps).txt}, from which 
we can initialize the gas. And then the orbital objects and icelines in order. (In the 
code, the orbital objects are called 'planet'). Then, we initialize the particles.
The particles from \textbf{Superparticles} objects is initially logrithmically 
distributed with the same physical sizes, and the total mass of particles is euqal to
the total solid mass get from initial gas surface density and dust-to-gas ratio. The 
object for particles are called 'Superparticles' because the particles in code 
are aggerates of many real particles, respectively, which help us dramatically 
save the computational power, also that the difference between Lagrangian code 
and N-body code.
After getting all the parts of system, we calculate the time 
step from \textbf{core.new$\_$timestep}, which we will introduce in detail later. 


\subsection{RUNNING}
After the initialization, it comes normal running. The running steps are 
shown in the dotted frame of figure \ref{fig:cpdcode}. The order is critical for running,
let's see them in detail.

\begin{enumerate}
    \item \textbf{Add planet (orbital objects)}. This step is done by \textbf{add$\_$planet} function 
        under \textbf{System} object.
        You can choose adding objects at a certain time and certain location or 
        add them from certain planetesimal forming mechanism (TBD). 
    \item \textbf{Update particles}. The particles is from \textbf{Superparticles} object, which 
        includes all the 'superparticles' in the disk. The main properties of particles 
        include location, physical mass and total mass. The 'total mass' is the sum 
        of the masses of all particles the 'superparticle' represents. 

        The updating is done by \textbf{update$\_$particles} under Syetem object, for 
        which the users need to supply how the particles grows in the 
        \textbf{dm$\_$dt} function under \textbf{userfun.py}. The drift of particles is 
        done by function \textbf{radial$\_$v} in \text{main/physics.py} as 
        \citep{Whipple1972,AdachiEtal1976,Weidenschilling1980}:
        \begin{align}
            v_r = -\frac{St}{1+St^2} \eta v_K 
        \end{align}
        in which the $St$ is stokes number, representing the areodynamical size of particles, 
        the $\eta$ is the ratio between the gas pressure gradient and gravity in disk, and 
        $v_K$ represents Kepler velocity. 

        The composition of particles are also considered, but in our code, the change of 
        composition happens only when cross the conrresponding iceline. 

        Also you will see the superparticles have many other properties like stokes number, 
        scale height, surface density and so on, which are auxiliary properties get from 
        disk object, and need to be used at certain points.
    \item \textbf{Advance iceline and planet}. 
        This step discribe the interaction between particles 
        and icelines/planets (orbital objects) when particles drift across the location of 
        icelines/planets. 

        The planets accretes materials from superparticles, decreasing 
        the total mass of superparticles, and increasing the mass of planets, also this 
        mass will change the planets' composition. Users can supply the analitical accretion 
        efficiency algorithm to function \textbf{epsilon$\_$PA} in \textbf{userfun.py}. And 
        also planets may migrates in disk, which is discribed by function \textbf{planet
        $\_$migration}. For iceline, the iceline will also move, because the change of 
        midplane temperature, which is discribed by function \textbf{$get\_iceline\_location$} 
        under ICELINE object. The \textbf{advance$\_$iceline} also has the task to 
        change the composition of particles because of evaporation. 
    \item \textbf{Post process}. After update the properties of every part, the \textbf{post$\_$process}
        will do some making up things. There's brief introduction aboout what is done under 
        this function (also a comment in the code):
        \begin{enumerate}
            \item add and remove particles which cross the inner edge and enter from 
                outer edge. 
            \item resample particle distribution to keep the number of superparticles 
                stable, which 
                has two algorithms can be chosen in \textbf{Test/users/parameters.py} by change 
                the value of 'resampleMode'. The two algorithms are 
                'Nplevel' and 'splitmerge', the former one stablize the number by adjust 
                the total mass of superparticles, while the later by spilt and merge 
                superparticles.
            \item update the particle state vector
            \item remove planets which are engulted by central body. 
            \item update mass of central body.
        \end{enumerate}
    \item \textbf{Query jumps}. 
        The step judge whether the jumps can happen, the jump and how to 
        judge jumps will be introduced by sec. \ref{sec:b}. If the jumps were judged 
        can happen, then forward the system by jump, otherwise by just adding the 
        time step into system's time. 

    \item \textbf{Get new time step}. Whether we do the jump or not, we need to get the time step 
        for next step. The \textbf{new$\_$timestep} is a complex function, inwhich we 
        compare several timescales and let the shortest one to be next time step. 
        And here we also get some 'milestones' for restricting the jump step. 

    \item \textbf{Back up data}. Back up the properties of current state for calculating the 
        timescales. 

    \item \textbf{Print and update log}. 
        Print some things to the terminal and update the log file under  
        \textbf{/Test/user/log/}
        This step is done by the \textbf{do$\_$stuff} in \textbf{userfun.py}, which 
        needs to be supplied by users. 

    \item Finally, all of these steps are in a while loop, when the time of system 
        is smaller then \textbf{tmax} parameter in \textbf{parameters.py}. 
\end{enumerate}

        
\subsection{FINAL}
When the \textbf{system.time} larger equal than the tmax in \textbf{parameters.py}, 
the 'final' parameter will be 'True', then the while loop 
will be stopped, and the system class will be stored as .pickle file. 

Under the \textbf{/Test/zxl$\_$test/userfun}, there's a 'data' class, which can 
store and process the data to plot some figures.


\section{JUMP}
\label{sec:b}
The jump is a typical programing method aiming at accelerating simulation. Under the 
princple that the evolution of system will enter into a quasi-static state, during which 
we can regard the rate of evolution of system properties as fixed values. For example, 
when the distribution of solid particles in the circumplanetary disk has a very long 
timescale, the accretion rate of satellites in this disk will not change significantly, 
then we don't need to calculate the accretion rate every step, which can be in other 
words 'jumped'. Of course implementing the 'jump' thing need to be cautious and well 
thought out. So Let's present you how we do the jump in detail.

To get how long we can jump over, 
we consider several characteristic timescales, including those related to satellite 
mass increase, satellite migration, iceline position changes, and mass inflow (specifically 
for CPDs). These 
timescales can be expressed as follows:
\begin{align}
    \tau_{P} = \frac{P_{new}}{(|P_{old}-P_{new}|)}\Delta t,
\end{align}
in which $P$ represents the kinds of properties of system, which in our CPD case are 
the location of icelines and the mass inflow rate. 

Also for some special preperties like mass growth of objects in disk, which is not 
continuous, cannot be calculated like that. For these kinds of properties, we 
logrithmically fit the points on which mass change, and get the evolving timescale 
from the slope. For these kinds of properties, we can get how the errors grow  
when we prelong the jump time, and we artifically supply a paramteter serving as 
the maximum error can be tolerated in jump, from which we can futher restrict the
jump time. The maximum jump time can be derived as:
\begin{align}
    t_{jump,max} = \frac{\eta m}{\sigma_{dm/dt} - \sigma_{max} dm/dt}
\end{align}
in which the $\sigma_{max}$ is the maximum error we artifically defined in 
\textbf{paramteters.py} under every user directory.

Then we can choose the minimum time between these timescales and maximum jump 
time, and restrict it futher by just simply multipling a pre-fractor that is 
samller than 1 in front 
of this time, finally we get the jump time: $t_{jump}$. Also, there are some 
points that cannot be jumpt over, which are called 'milestones', like the time 
we inserting the objects. Comparing $t_{jump}$ and the time to milestones, 
we finally get the time we can jump over. 

After we get how long we can jump over, we also need to know whether we can jump, 
which is critical because if we do the jump very frequently, we can imagine the 
error will be large. So we consider the conditions below:
\begin{enumerate}
    \item The jump step size must be significantly larger than the evolution step size to 
        ensure its significance.
    \item The jump step size must be greater than 100 times average steps.
    \item Jumps cannot occur continuously; there must be sufficient time between jumps to 
        update various physical processes’ evolution rates. For instance, we currently 
        require at least 100 steps of regular evolution between jumps.
    \item Jumps happen only after the system enter into the quasi-static state. We 
        do this by letting all the initially distributed particles are accreted 
        by the central body in our CPD cases. (NOT SUITABLE FOR PPD)
\end{enumerate}

Only when all these conditions are met can a jump occur. Figure \ref{fig:jumpT} 
illustrates the jump durations in a simulation with $10^7$ years, in which 
we can found:
\begin{enumerate}
    \item Throughout the simulation, significant jumps occur, greatly reducing the required computation time.
    \item The time spent on regular equation solving and other steps between jumps is minimal.
    \item Jumps halt when satellites enter the system.
    \item Sharp decreases in jump duration correspond to satellite capture into resonances, 
        as expected. Since we cannot precisely predict when satellites will enter resonances, 
        we reduce jump time near resonance points to avoid skipping them.
\end{enumerate}

\begin{figure}[t]
    \centering
    \includegraphics *[width=0.8\textwidth]{figure/jumpT.jpg}
    \caption{The jump steps through the simulation. The blue line with dots represents 
    the length of jump steps. The black dashed line represents the time points 
    where the satellites enter into the system. The slim dotted lines represents the 
    ends of every times, which is almost overlap with the beginning of 
    next jump.}
    \label{fig:jumpT}
\end{figure}


\bibliography{ads}
\bibliographystyle{aa}

\end{document}
